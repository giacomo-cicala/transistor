\documentclass[12pt,italian]{article}
\usepackage[italian]{babel}
\usepackage[a4paper, margin=1.97cm]{geometry}
\usepackage{graphicx}
\usepackage{amsmath}
\usepackage{circuitikz}
\usetikzlibrary{calc}
\usepackage{caption}
\usepackage{subcaption}
\usepackage{amsfonts}
\usepackage[hidelinks]{hyperref}
\usepackage{cleveref}
\usepackage{siunitx}
\sisetup{
	table-number-alignment = center,
	table-figures-integer = 4,
	table-figures-decimal = 8
}
\usepackage{booktabs}
\graphicspath{{./images/}}

\newcommand{\err}[1]{\textcolor{red}{#1}}
\crefname{table}{tab.}{tab.}

\title{Misura della caratteristica di uscita di un BJT P-N-P in configurazione a emettitore comune}
\author{Enrico Barbuio \\ 0001117553 \and Giacomo Cicala \\ 0001122965} 
\date{Turno 1 del 28 Novembre 2025, Tavolo 11}

\begin{document}
\maketitle

\renewcommand{\abstractname}{Abstract}
\begin{abstract}
	Sono state svolte delle misure su un transistor BJT PNP in configurazione emettitore comune.
	In particolare: è stata misurata la tensione di Early:
	\[
		V_A =
	\]
	ed è stato stimato il fattore di amplificazione:
	\[
		\beta =
	\]
\end{abstract}

\section*{Introduzione}

L'obiettivo dell'esperimento è la misura di tenisone di Early e fattore di
amplificazione di un transistor PNP in configurazione emettitore comune.

Data la caratteristica I-V del collettore con corrente di base fissata, nella
regione attiva ci si aspetta un andamento lineare. Estendendo questa retta
verso l'asse delle tensioni si avrà un intersezione nel punto $V_a$, detto
tensione di Early. Scrivendo la retta come:
\begin{equation}
	\label{eq:VA}
	V _{CE} = a + b I_{C}
\end{equation}
la tensione di Early non è altro che l'ordinata all'origine: $V_a = a$. Il
parametro $b$ invece rappresenta la resistenza del ramo tra emettitore e
collettore con $I_b$ fissato (\err{nome specifico?}). La conduttanza, definita
come:
\begin{equation}
	\label{eq:conduttanza}
	g = \frac{\Delta I_C}{\Delta V_{CE}}
\end{equation}
sarà il reciproco di $b$.
Variando la corrente di base e misurando nuovamente la corrente del collettore
alla stessa tensione, si può ottenere una stima del guadagno di corrente:
\begin{equation}
	\label{eq:beta}
	\beta = \frac{\Delta I_{C}}{\Delta I_{B}}
\end{equation}
dove $\Delta I_{B}$ è la differenza tra le due correnti di base
e $\Delta I_{C}$ è la differenza tra le correnti di collettore corrispondenti
alla stessa tensione di collettore.

\section*{Apparato sperimentale e svolgimento}
\begin{figure}
	\centering
	\begin{circuitikz}
		% POWER RAILS
		\draw (0,0.5) -- (0,10) node[above] {\textbf{GND}};
		\draw (1,0.5) -- (1,10) node[above] {\textbf{-5V}};

		% TOP POTENTIOMETER
		\draw (0,9) to[short, *-] ++(2,0) to [pR, name=pot_c, l_=$R_C$] ++(0,-2) to[short, -*] ++(-1,0);

		% TRANSISTOR (PNP)
		\draw (4, 4.5) node[pnp, rotate=180, tr circle, anchor=center] (T) {};
		\node[right] at (T.emitter) {E};
		\node[above right] at (T.base) {B};
		\node[right] at (T.collector) {C};
		% Emitter
		\draw (T.emitter) -- (0, 0 |- T.emitter) node[circ]{};
		% Base
		\draw (T.base) -- ++(1,0) coordinate (base);
		% Collector
		\draw (T.collector) -- (T.collector |- pot_c.wiper) -- (pot_c.wiper);

		% BOTTOM POTENTIOMETER
		\draw (0,3) to[short, *-] ++(2,0) to [pR, name=pot_b, l_=$R_B$] ++(0,-2) to[short, -*] ++(-1,0);
		\draw (base) to [rmeterwa, t=A, i=$I_B < 0$] (pot_b.wiper -| base) -- (pot_b.wiper);
	\end{circuitikz}
	\qquad
	\begin{circuitikz}
		% POWER RAILS
		\draw (0,0.5) -- (0,10) node[above] {\textbf{GND}};
		\draw (1,0.5) -- (1,10) node[above] {\textbf{-5V}};

		% TOP POTENTIOMETER
		\draw (0,9) to[short, *-] ++(2,0) to [pR, name=pot_c, l_=$R_C$] ++(0,-2) to[short, -*] ++(-1,0);

		% TRANSISTOR (PNP)
		\draw (4, 4.5) node[pnp, rotate=180, tr circle, anchor=center] (T) {};
		\node[right] at (T.emitter) {E};
		\node[above right] at (T.base) {B};
		\node[right] at (T.collector) {C};
		% Emitter
		\draw (T.emitter) -- (0, 0 |- T.emitter) node[circ]{};
		% Base
		\draw (T.base) -- ++(1,0) coordinate (base);
		% Collector
		\draw (T.collector) to [short, -*] ++(0,0.5) coordinate (collector) to [rmeterwa, t=A, i_=$I_C < 0$] (T.collector |- pot_c.wiper) -- (pot_c.wiper);
		\draw (collector -| 0,0) to[short,*-] ++(1.5,0) coordinate (tmp) to [rmeterwa, t=V, v=$V_{CE}$] (collector);

		% BOTTOM POTENTIOMETER
		\draw (0,3) to[short, *-] ++(2,0) to [pR, name=pot_b, l_=$R_B$] ++(0,-2) to[short, -*] ++(-1,0);
		\draw (pot_b.wiper) -- (pot_b.wiper -| base) -- (base);
	\end{circuitikz}\\
	\err{sistema verso tensione e corrente}
	\caption{Circuito con transistor PNP in configurazione a emettitore comune.
		A sinistra è raffigurato lo schema per la misura della corrente di base.
		A destra quello per la misura di corrente e tensione del collettore.}
	\label{fig:circuito}
\end{figure}
\begin{figure}[h!]
	\centering
	\begin{subfigure}[b]{0.32\textwidth}
		\centering
		\includegraphics[width=\linewidth]{images/alimentatore.png}
		\caption{Alimentatore}
		\label{fig:alimentatore}
	\end{subfigure}
	\hfill
	\begin{subfigure}[b]{0.32\textwidth}
		\centering
		\includegraphics[width=\linewidth]{images/oscilloscopio3.png}
		\caption{Oscilloscopio}
		\label{fig:oscilloscopio}
	\end{subfigure}
	\hfill
	\begin{subfigure}[b]{0.32\textwidth}
		\centering
		\includegraphics[width=\linewidth]{images/multimetro.png}
		\caption{Multimetro}
		\label{fig:multimetro}
	\end{subfigure}
	\caption{Strumentazione usata nell'apparato sperimentale:
		(a) Alimentatore  \texttt{TTi EB2025T}; (b) Oscilloscopio
		\texttt{GW Instek GOS-652G}; (c) Multimetro
		\texttt{Fluke 175 True RMS Multimeter}.}
	\label{fig:apparato_sperimentale}
\end{figure}

\'E stato realzzato il circuito di \cref{fig:circuito} con i seguenti componenti:
\begin{itemize}
	\item Transistor PNP:\@ \texttt{N3906(BU)}
	\item Potenziometri:\@ $R_B = \qty{100}{\kohm}$, $R_C = \qty{10}{\kohm}$
\end{itemize}

Il circuito è alimentato da un generatore a tensione costante (modello
\texttt{TTi EB2025T}, \cref{fig:alimentatore}) impostato a \qty{-5}{\V}. I due
potenziometri funzionano da partitori di tensione, comportandosi come due
sensibili generatori di tensione variabile che ci permettono di impostare le
correnti di base e collettore. Le misure di tensione sono state svolte con un
oscilloscopio da banco \texttt{GW Instek GOS-652G} (\cref{fig:oscilloscopio}).
Per le misure di corrente si è utilizzato un multimetro \texttt{Fluke 175 True
	RMS Multimeter} (\cref{fig:multimetro}).

Come prima cosa è stata impostata la corrente di base. Realizzato il primo
circuito di \cref{fig:circuito}, si è variato il potenziometro $R_B$ fino a
raggiungere \qty{50}{\uA} sulla base.

Dopodiché lasciando invariato $R_B$ è stato portato il circuito nella seconda
configurazione per misurare $V_{CE}$ e $I_C$. Cambiando ora la resistenza del
potenziometro del collettore, sono stati misurati diversi punti di tensione e
corrente nel range \qty{0}{\V} -- \qty{4}{\V}.

Lo stesso procedimento è stato ripetuto con corrente di base $I_B =
	\qty{100}{\uA}$.

\section*{Risultati e discussione}

Si riportano in (\cref{tab:50uA}) e in (\cref{tab:100uA}) le misure delle
coppie di valori I-V rispettivamente per corrente di base di $\qty{-50}{\uA}$ e
\qty{-100}{\uA}. I valori di tensione e corrente tabulati sono già stati
cambiati di segno per ottenere le caratteristiche nel primo quadrante. Per il
calcolo degli errori associati alle misure di tensione e corrente si rimanda
all'appendice \ref{appendice:errori_I-V}. Le caratteristiche di uscita del
transistor per le due correnti di base considerate sono riportate in
(\cref{fig:caratteristiche}).

\begin{table}[hbtp]
	\centering
	% --- INIZIO PRIMA META' (SINISTRA) ---
	\begin{minipage}[t]{0.48\textwidth}
		\vspace{0pt}
		\centering
		\begin{tabular}[t]{
				S[table-format=1.3]
				S[table-format=2.2]
				S[table-format=1.3]
				S[table-format=1.2] % Errore corrente: 2 decimali
				S[table-format=1.1]
			}
			\toprule
			{ \boldmath $-V_{CE}$ } & { \boldmath $-I_{C}$ } & { \boldmath $\sigma_{V}$ } & { \boldmath $\sigma_{I}$ } & { \boldmath $F.S._{osc}$ } \\
			{ (V) }                 & { (mA) }               & { (V) }                    & { (mA) }                   & { (V/div) }                \\
			\midrule
			4.00                    & 10.97                  & 0.16                       & 0.14                       & 1                          \\
			3.80                    & 10.92                  & 0.15                       & 0.14                       & 1                          \\
			3.60                    & 10.86                  & 0.15                       & 0.14                       & 1                          \\
			3.40                    & 10.80                  & 0.14                       & 0.14                       & 1                          \\
			3.20                    & 10.73                  & 0.14                       & 0.14                       & 1                          \\
			3.00                    & 10.62                  & 0.10                       & 0.14                       & 0.5                        \\
			2.80                    & 10.53                  & 0.10                       & 0.14                       & 0.5                        \\
			2.60                    & 10.46                  & 0.09                       & 0.14                       & 0.5                        \\
			2.40                    & 10.39                  & 0.09                       & 0.13                       & 0.5                        \\
			2.20                    & 10.32                  & 0.08                       & 0.13                       & 0.5                        \\
			2.00                    & 10.24                  & 0.08                       & 0.13                       & 0.5                        \\
			1.80                    & 10.14                  & 0.07                       & 0.13                       & 0.5                        \\
			1.60                    & 10.05                  & 0.07                       & 0.13                       & 0.5                        \\
			1.40                    & 9.97                   & 0.07                       & 0.13                       & 0.5                        \\
			1.20                    & 9.86                   & 0.06                       & 0.13                       & 0.5                        \\
			\bottomrule
		\end{tabular}
	\end{minipage}
	\hfill
	% --- INIZIO SECONDA META' (DESTRA) ---
	\begin{minipage}[t]{0.48\textwidth}
		\vspace{0pt}
		\centering
		\begin{tabular}[t]{
				S[table-format=1.3]
				S[table-format=2.2]
				S[table-format=1.3]
				S[table-format=1.2] % Errore corrente: 2 decimali
				S[table-format=1.1]
			}
			\toprule
			{ \boldmath $-V_{CE}$ } & { \boldmath $-I_{C}$ } & { \boldmath $\sigma_{V}$ } & { \boldmath $\sigma_{I}$ } & { \boldmath $F.S._{osc}$ } \\
			{ (V) }                 & { (mA) }               & { (V) }                    & { (mA) }                   & { (V/div) }                \\
			\midrule
			1.00                    & 9.75                   & 0.04                       & 0.13                       & 0.2                        \\
			0.80                    & 9.62                   & 0.03                       & 0.13                       & 0.2                        \\
			0.60                    & 9.49                   & 0.03                       & 0.13                       & 0.2                        \\
			0.52                    & 9.41                   & 0.03                       & 0.12                       & 0.2                        \\
			0.44                    & 9.34                   & 0.02                       & 0.12                       & 0.2                        \\
			0.400                   & 9.27                   & 0.016                      & 0.12                       & 0.1                        \\
			0.360                   & 9.22                   & 0.015                      & 0.12                       & 0.1                        \\
			0.320                   & 9.15                   & 0.014                      & 0.12                       & 0.1                        \\
			0.280                   & 8.99                   & 0.013                      & 0.12                       & 0.1                        \\
			0.240                   & 8.63                   & 0.012                      & 0.12                       & 0.1                        \\
			0.200                   & 7.77                   & 0.012                      & 0.11                       & 0.1                        \\
			0.180                   & 7.08                   & 0.011                      & 0.10                       & 0.1                        \\
			0.160                   & 5.97                   & 0.011                      & 0.09                       & 0.1                        \\
			0.140                   & 4.81                   & 0.011                      & 0.08                       & 0.1                        \\
			\\ % Riga vuota per pareggiare la lunghezza
			\bottomrule
		\end{tabular}
	\end{minipage}
	\caption{Misure di tensione e corrente per $I_{B} = \qty{-50}{\uA}$. La prima colonna riporta le tensioni collettore-emettitore, la seconda le correnti di collettore, la terza gli errori sulla misure di tensione, la quarta gli errori sulle misure di corrente, la quinta il fondo scala dell'oscilloscopio.}
	\label{tab:50uA}

\end{table}

\begin{table}[h]
	\centering
	% --- META' SINISTRA (Righe 1-32) ---
	\begin{minipage}[t]{0.48\textwidth}
		\vspace{0pt}
		\centering
		\begin{tabular}[t]{
				S[table-format=1.3]
				S[table-format=2.1] % Tutte arrotondate a 1 decimale per coerenza con l'errore 0.2
				S[table-format=1.3]
				S[table-format=1.1] % Errore fisso a 0.2
				S[table-format=1.1]
			}
			\toprule
			{ \boldmath $-V_{CE}$ } & { \boldmath $-I_{C}$ } & { \boldmath $\sigma_{V}$ } & { \boldmath $\sigma_{I}$ } & { \boldmath $F.S._{osc}$ } \\
			{ (V) }                 & { (mA) }               & { (V) }                    & { (mA) }                   & { (V/div) }                \\
			\midrule
			4.00                    & 20.7                   & 0.16                       & 0.2                        & 1                          \\
			3.80                    & 20.7                   & 0.15                       & 0.2                        & 1                          \\
			3.60                    & 20.6                   & 0.15                       & 0.2                        & 1                          \\
			3.40                    & 20.4                   & 0.14                       & 0.2                        & 1                          \\
			3.20                    & 20.3                   & 0.14                       & 0.2                        & 1                          \\
			3.00                    & 20.2                   & 0.13                       & 0.2                        & 1                          \\
			2.90                    & 19.9                   & 0.10                       & 0.2                        & 0.5                        \\
			2.80                    & 19.8                   & 0.10                       & 0.2                        & 0.5                        \\
			2.70                    & 19.7                   & 0.10                       & 0.2                        & 0.5                        \\
			2.60                    & 19.6                   & 0.09                       & 0.2                        & 0.5                        \\
			2.50                    & 19.5                   & 0.09                       & 0.2                        & 0.5                        \\
			2.40                    & 19.4                   & 0.09                       & 0.2                        & 0.5                        \\
			2.30                    & 19.3                   & 0.09                       & 0.2                        & 0.5                        \\
			2.20                    & 19.2                   & 0.08                       & 0.2                        & 0.5                        \\
			2.10                    & 19.1                   & 0.08                       & 0.2                        & 0.5                        \\
			2.00                    & 19.0                   & 0.08                       & 0.2                        & 0.5                        \\
			1.90                    & 18.9                   & 0.08                       & 0.2                        & 0.5                        \\
			1.80                    & 18.8                   & 0.07                       & 0.2                        & 0.5                        \\
			1.70                    & 18.7                   & 0.07                       & 0.2                        & 0.5                        \\
			1.60                    & 18.6                   & 0.07                       & 0.2                        & 0.5                        \\
			1.50                    & 18.5                   & 0.07                       & 0.2                        & 0.5                        \\
			1.40                    & 18.4                   & 0.07                       & 0.2                        & 0.5                        \\
			1.30                    & 18.3                   & 0.06                       & 0.2                        & 0.5                        \\
			1.20                    & 18.2                   & 0.06                       & 0.2                        & 0.5                        \\
			1.10                    & 18.0                   & 0.06                       & 0.2                        & 0.5                        \\
			1.00                    & 17.9                   & 0.04                       & 0.2                        & 0.2                        \\
			0.92                    & 17.8                   & 0.03                       & 0.2                        & 0.2                        \\
			0.84                    & 17.7                   & 0.03                       & 0.2                        & 0.2                        \\
			0.76                    & 17.6                   & 0.03                       & 0.2                        & 0.2                        \\
			0.68                    & 17.5                   & 0.03                       & 0.2                        & 0.2                        \\
			0.60                    & 17.3                   & 0.03                       & 0.2                        & 0.2                        \\
			0.52                    & 17.1                   & 0.03                       & 0.2                        & 0.2                        \\
			\bottomrule
		\end{tabular}
	\end{minipage}
	\hfill
	% --- META' DESTRA (Righe 33-64) ---
	\begin{minipage}[t]{0.48\textwidth}
		\vspace{0pt}
		\centering
		\begin{tabular}[t]{
				S[table-format=1.3]
				S[table-format=2.2] % Misto: 2 decimali tranne una riga
				S[table-format=1.3]
				S[table-format=1.2] % Misto: 2 decimali tranne una riga
				S[table-format=1.1]
			}
			\toprule
			{ \boldmath $-V_{CE}$ } & { \boldmath $-I_{C}$ } & { \boldmath $\sigma_{V}$ } & { \boldmath $\sigma_{I}$ } & { \boldmath $F.S._{osc}$ } \\
			{ (V) }                 & { (mA) }               & { (V) }                    & { (mA) }                   & { (V/div) }                \\
			\midrule
			0.44                    & 16.8                   & 0.02                       & 0.2                        & 0.2                        \\
			0.40                    & 16.6                   & 0.02                       & 0.2                        & 0.2                        \\
			0.380                   & 16.10                  & 0.015                      & 0.19                       & 0.1                        \\
			0.340                   & 15.85                  & 0.014                      & 0.19                       & 0.1                        \\
			0.320                   & 15.60                  & 0.014                      & 0.19                       & 0.1                        \\
			0.300                   & 15.28                  & 0.013                      & 0.19                       & 0.1                        \\
			0.290                   & 15.05                  & 0.010                      & 0.19                       & 0.05                       \\
			0.280                   & 14.85                  & 0.010                      & 0.18                       & 0.05                       \\
			0.270                   & 14.60                  & 0.010                      & 0.18                       & 0.05                       \\
			0.260                   & 14.35                  & 0.009                      & 0.18                       & 0.05                       \\
			0.250                   & 14.04                  & 0.009                      & 0.18                       & 0.05                       \\
			0.240                   & 13.65                  & 0.009                      & 0.17                       & 0.05                       \\
			0.230                   & 13.27                  & 0.009                      & 0.17                       & 0.05                       \\
			0.220                   & 12.80                  & 0.008                      & 0.17                       & 0.05                       \\
			0.210                   & 12.30                  & 0.008                      & 0.17                       & 0.05                       \\
			0.200                   & 11.70                  & 0.008                      & 0.16                       & 0.05                       \\
			0.190                   & 11.07                  & 0.008                      & 0.16                       & 0.05                       \\
			0.180                   & 10.30                  & 0.007                      & 0.15                       & 0.05                       \\
			0.170                   & 9.49                   & 0.007                      & 0.15                       & 0.05                       \\
			0.160                   & 8.62                   & 0.007                      & 0.14                       & 0.05                       \\
			0.150                   & 7.75                   & 0.007                      & 0.14                       & 0.05                       \\
			0.140                   & 6.75                   & 0.007                      & 0.10                       & 0.05                       \\
			0.130                   & 5.76                   & 0.006                      & 0.09                       & 0.05                       \\
			0.120                   & 4.87                   & 0.006                      & 0.08                       & 0.05                       \\
			0.110                   & 4.07                   & 0.006                      & 0.07                       & 0.05                       \\
			0.100                   & 3.27                   & 0.004                      & 0.06                       & 0.02                       \\
			0.092                   & 2.68                   & 0.003                      & 0.06                       & 0.02                       \\
			0.084                   & 2.13                   & 0.003                      & 0.05                       & 0.02                       \\
			0.076                   & 1.68                   & 0.003                      & 0.05                       & 0.02                       \\
			0.068                   & 1.29                   & 0.003                      & 0.04                       & 0.02                       \\
			0.060                   & 0.96                   & 0.003                      & 0.04                       & 0.02                       \\
			\\
			\bottomrule
		\end{tabular}
	\end{minipage}
	\caption{Misure di tensione e corrente per $I_{B} = \qty{-100}{\uA}$. La prima colonna riporta le tensioni collettore-emettitore, la seconda le correnti di collettore, la terza gli errori sulla misure di tensione, la quarta gli errori sulle misure di corrente, la quinta il fondo scala dell'oscilloscopio.}
	\label{tab:100uA}
\end{table}

\begin{figure}[hbtp]
	\centering
	\includegraphics[width=0.7\textwidth]{caratteristiche.pdf}
	\caption{Caratteristiche di uscita del transistor per le correnti di base
		$I_{B}= \qty{-50}{\uA}$ (in blu) e $I_{B}= \qty{-100}{\uA}$ (in rosso)
		con i relativi fit lineari nella regione attiva $1$ -- $3.5$ \unit{\V}.
		Sulle ascisse sono riportate le tensioni collettore-emettitore, sulle
		ordinate le correnti di collettore.}
	\label{fig:caratteristiche}
\end{figure}

Per ottenere delle stime per la tensione di Early e per la conduttanza, si sono
eseguiti dei fit lineari come in \cref{eq:VA} nella regione attiva del
transistor, compresa tra \qty{1}{\V} e \qty{3.5}{\V} V. I valori dei parametri di fit
ottenuti per le due correnti di base sono:
\begin{equation*}
	a_{50} = (22 \pm 3) \ \unit{\V} \hspace{2cm} b_{50} = (2.3 \pm 0.3) \ \unit{\kohm}
\end{equation*}
\begin{equation*}
	a_{100} = (16.0 \pm 1.3) \ \unit{\V} \hspace{2cm} b_{100} = (0.95 \pm 0.07) \ \unit{\kohm}
\end{equation*}
Dunque, per le considerazioni già fatte, abbiamo che le due stime per la tensione di Early e per la conduttanza sono
\begin{equation*}
	V_{a,50} = (22 \pm 3) \ \unit{\V} \hspace{2cm} V_{a,100} = (16.0 \pm 1.3) \ \unit{\V}
\end{equation*}
\begin{equation*}
	g_{50} = (0.43 \pm 0.05) \ \unit{\kohm} \hspace{2cm} g_{100} = (1.05 \pm 0.07) \ \unit{\kohm}
\end{equation*}

\noindent
le cui incertezze sono state calcolate come descritto nella \cref{appendice:errori_conduttanza}.

Tramite l'\cref{eq:beta}, scegliendo come tensione di collettore $-V_{CE} =
	\qty{3}{\V}$, si ottiene una stima del guadagno di corrente del transistor pari
a
\begin{equation*}
	\beta = 190.6
\end{equation*}

Per quanto riguarda gli andamenti delle caratteristiche di uscita del
transistor, si nota come per entrambe le correnti di base considerate, queste
rispecchino qualitativamente la forma prevista dalle equazioni di Ebers-Moll
per il BJT in configurazione a emettitore comune. Inoltre, si osserva che, in
regione attiva, all'aumentare della corrente di base, aumenta linearmente anche
la corrente di collettore, confermando l'effetto Early. Tuttavia, si nota come
i valori della tensione di Early ottenuti dai fit lineari per le due correnti
di base considerate non siano compatibili tra loro entro l'incertezza
sperimentale. Nonostante ciò, il valore della conduttanza per la corrente di
base di \qty{-100}{\uA} risulta maggiore rispetto a quella di \qty{-50}{\uA},
\err{compatibilmente col fatto che la tensione di Early debba essere unica per
	tutte le correnti, e che dunque la pendenza della retta della caratteristica in
	regione attiva debba essere maggiore per correnti più alte.}
\section*{Conclusioni}

I valori della tensione di Early ottenuti per le due correnti di base a
\qty{-50}{\uA} e a \qty{-100}{\uA} sono
\begin{equation*}
	V_{a,50} = (22 \pm 3) \ \unit{\V} \hspace{2cm} V_{a,100} = (16.0 \pm 1.3) \ \unit{\V}
\end{equation*}
quindi risultano non compatibili tra loro entro l'incertezza sperimentale.

Le stime delle conduttanze per le due correnti di base sono
\begin{equation*}
	g_{50} = (0.43 \pm 0.05) \ \unit{\kohm} \hspace{2cm} g_{100} = (1.05 \pm 0.07) \ \unit{\kohm}
\end{equation*}

Invece, la stima del guadagno di corrente del transistor ottenuta scegliendo
come tensione di collettore $-V_{CE} = \qty{3}{\V}$ è pari a
\begin{equation*}
	\beta = 190.6
\end{equation*}

\appendix
\section{Appendici}
\subsection{Calcolo degli errori per tensioni e correnti}
\label{appendice:errori_I-V}

La risoluzione dell'oscilloscopio (e quindi l'errore sulla lettura)
$\sigma_{l}$ è stata calcolata come
\begin{equation}
	\sigma_{l} = \frac{\text{F.S.}}{5}*(\text{\#tacchette apprezzabili})
\end{equation}
dove in questo caso, in tutte le misure dell'esperimento, il numero di
tacchette apprezzabili è stato 0.5. L'errore totale associato ad ogni misura di
tensione con l'oscilloscopio è stato calcolato come
\begin{equation}
	\sigma_{V} = \sqrt{(\sigma_{l})^{2} + (\sigma_{c})^{2}}
\end{equation}

\noindent
dove l'errore del costruttore è

\begin{equation}
	\frac{\sigma_{c}}{V_{mis}} = 3\%
\end{equation}

\noindent
con $V_{mis}$ tensione misurata; l'errore sullo zero dell'oscilloscopio è stato
omesso in quanto si è verificato essere trascurabile rispetto agli altri
errori, grazie ad un opportuno fondo scala di $5$mV/Div. Invece, l'errore sulla
misura di corrente è stato calcolato considerando l'errore del multimetro, che,
per il fondo scala utilizzato per tutte le misure $F.S._{mult} = 60$mA, è dato
da
\begin{equation}
	\frac{\sigma_{I}}{I_{mis}} = 1.5\% + 3 \text{ digits}
\end{equation}

\subsection{Calcolo dell'errore sulla tensione di Early}
\label{appendice:errori_conduttanza}
L'errore sulla conduttanza $g$ è stato calcolato tramite propagazione degli
errori a partire dall'\cref{eq:conduttanza} come
\begin{equation}
	\sigma_{g} = \frac{\sigma_b}{b^2}
\end{equation}

\end{document}