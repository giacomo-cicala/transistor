\documentclass[12pt,italian]{article}
\usepackage[italian]{babel}
\usepackage[a4paper, margin=1.97cm]{geometry}
\usepackage{graphicx}
\usepackage{amsmath}
\usepackage{circuitikz}
\usepackage{caption}
\usepackage{subcaption}
\usepackage{amsfonts}
\usepackage[hidelinks]{hyperref}
\usepackage{cleveref}
\usepackage{siunitx}
\usepackage{booktabs}
\graphicspath{{./images/}}

\newcommand{\err}[1]{\textcolor{red}{#1}}
\crefname{table}{tab.}{tab.}

\title{Misura della caratteristica di uscita di un BJT P-N-P in configurazione a emettitore comune}
\author{Enrico Barbuio \\ 0001117553 \and Giacomo Cicala \\ 0001122965} 
\date{\today}

\begin{document}
\maketitle

\renewcommand{\abstractname}{Abstract}
\begin{abstract}
prova
\end{abstract}

\section*{Introduzione}

\section*{Apparato sperimentale e svolgimento}

\section*{Risultati e discussione}
Si riportano in (\cref{tab:50uA}) e in (\cref{tab:100uA}) le misure delle coppie di valori I-V rispettivamente per corrente di base di 50$\mu$A e 100$\mu$A.

\sisetup{
    table-number-alignment = center,
    table-figures-integer = 4,
    table-figures-decimal = 8
}

\begin{table}[h]
\centering
\begin{tabular}{
    S[table-format=1.4] % 4 decimali per il multimetro (es. 0.0568)
    S[table-format=1.3] % 3 decimali max per l'oscilloscopio
    S[table-format=1.2] % Scala
    S[table-format=1.3] % Risoluzione
    S[table-format=1.3] % Errore (varia tra 2 e 3 decimali)
}
\toprule
{ \boldmath $V$ } &
{ \boldmath $I$ } &
{ \boldmath $F.S._{osc}$ } &
{ \boldmath $\sigma_{V}$ } &
{ \boldmath $\sigma_{I}$ } \\

{ (V) } & { (mA) } & { (V/div) } & { (V) } & { (mA) } \\
\midrule
0.0568 & 0.052 & 0.02 & 0.002 & 0.003 \\
0.1453 & 0.140 & 0.05 & 0.005 & 0.007 \\
0.2288 & 0.220 & 0.05 & 0.005 & 0.008 \\
0.2980 & 0.300 & 0.10 & 0.010 & 0.013 \\
0.3773 & 0.380 & 0.10 & 0.010 & 0.015 \\
0.4537 & 0.460 & 0.10 & 0.010 & 0.017 \\
0.5316 & 0.540 & 0.10 & 0.010 & 0.019 \\
0.6350 & 0.64  & 0.20 & 0.020 & 0.03 \\ 
0.7090 & 0.72  & 0.20 & 0.020 & 0.03 \\ 
0.8150 & 0.80  & 0.20 & 0.020 & 0.03 \\
\bottomrule
\end{tabular}
    
\caption{Misure di tensione e corrente per $I_{B} = 50 \mu A$. \err{SISTEMARE I VALORI CORRETTI}}
\label{tab:50uA}
\end{table}

\begin{table}[h]
\centering
\begin{tabular}{
    S[table-format=1.4] % 4 decimali per il multimetro (es. 0.0568)
    S[table-format=1.3] % 3 decimali max per l'oscilloscopio
    S[table-format=1.2] % Scala
    S[table-format=1.3] % Risoluzione
    S[table-format=1.3] % Errore (varia tra 2 e 3 decimali)
}
\toprule
{ \boldmath $V$ } &
{ \boldmath $I$ } &
{ \boldmath $F.S._{osc}$ } &
{ \boldmath $\sigma_{V}$ } &
{ \boldmath $\sigma_{I}$ } \\

{ (V) } & { (mA) } & { (V/div) } & { (V) } & { (mA) } \\
\midrule
0.0568 & 0.052 & 0.02 & 0.002 & 0.003 \\
0.1453 & 0.140 & 0.05 & 0.005 & 0.007 \\
0.2288 & 0.220 & 0.05 & 0.005 & 0.008 \\
0.2980 & 0.300 & 0.10 & 0.010 & 0.013 \\
0.3773 & 0.380 & 0.10 & 0.010 & 0.015 \\
0.4537 & 0.460 & 0.10 & 0.010 & 0.017 \\
0.5316 & 0.540 & 0.10 & 0.010 & 0.019 \\
0.6350 & 0.64  & 0.20 & 0.020 & 0.03 \\ 
0.7090 & 0.72  & 0.20 & 0.020 & 0.03 \\ 
0.8150 & 0.80  & 0.20 & 0.020 & 0.03 \\
\bottomrule
\end{tabular}
    
\caption{Misure di tensione e corrente per $I_{B} = 100 \mu A$. \err{SISTEMARE I VALORI CORRETTI}}
\label{tab:100uA}
\end{table}


La risoluzione dell'oscilloscopio (e quindi l'errore sulla lettura) $\sigma_{l}$ è stata calcolata come
\begin{equation}
    \sigma_{l} = \frac{F.S}{5}*(\text{\#tacchette apprezzabili})
\end{equation}
dove in questo caso, in tutte le misure dell'esperimento, il numero di tacchette apprezzabili è stato 0.5. L'errore totale associato ad ogni misura di tensione con l'oscilloscopio è stato calcolato come

\begin{equation}
    \sigma_{V} = \sqrt{(\sigma_{l})^{2} + (\sigma_{c})^{2}}
\end{equation}

\noindent
dove l'errore del costruttore è

\begin{equation}
    \frac{\sigma_{c}}{V_{mis}} = 3\%
\end{equation} 

\noindent
con $V_{mis}$ tensione misurata; l'errore sullo zero dell'oscilloscopio è stato trascurato in quanto si è verificato che fosse trascurabile rispetto agli altri errori, grazie ad un opportuno fondo scala di $5$mV/Div. Invece, l'errore sulla misura di corrente è stato calcolato considerando l'errore del multimetro, che, per il fondo scala utilizzato per tutte le misure $F.S._{mult} = 60$mA, è dato da
\begin{equation}
    \frac{\sigma_{I}}{I_{mis}} = 1.5\% + 3 \text{ digits}
\end{equation}

Dalle caratteristiche di uscita del transistor per le due correnti di base riportate in \cref{fig:caratteristiche}, si sono eseguiti due fit lineari nella regione attiva del transistor (\err{scegli})
\begin{equation}
    I= a+bV
\end{equation}

\noindent
per ottenere il valore della tensione di Early

\begin{equation}
    V_{A} = -\frac{a}{b}
\end{equation}

\begin{figure}[h!]
    \centering
    \includegraphics[width=0.7\textwidth]{example-image}
    \caption{Fit lineare delle caratteristiche di uscita del transistor per $I_{B} = 50 \mu A$ (\err{colore}) e per $I_{B} = 100 \mu A$ (\err{colore}). nella regione attiva (\err{scegli}).}
    \label{fig:fit_calibrazione}
\end{figure}
Si può ottenere inoltre una stima (\err{senza incertezza}) del guadagno di corrente

\begin{equation}
    \beta = \frac{\Delta I_{C}}{\Delta I_{B}}
\end{equation}

\noindent
dove $\Delta I_{C}$ è la differenza tra le correnti di collettore misurate per le due correnti di base, alla stessa tensione di collettore (nel nostro caso $V_{CE} = 3V$), e $\Delta I_{B}$ è la differenza tra le due correnti di base.





\section*{Conclusioni}

\appendix
\section{Appendici}

\end{document}