\documentclass[12pt,italian]{article}
\usepackage[italian]{babel}
\usepackage[a4paper, margin=1.97cm]{geometry}
\usepackage{graphicx}
\usepackage{amsmath}
\usepackage{circuitikz}
\usepackage{caption}
\usepackage{subcaption}
\usepackage{amsfonts}
\usepackage[hidelinks]{hyperref}
\usepackage{cleveref}
\usepackage{siunitx}
\usepackage{booktabs}
\graphicspath{{./images/}}

\newcommand{\err}[1]{\textcolor{red}{#1}}
\crefname{table}{tab.}{tab.}

\title{Misura della caratteristica di uscita di un BJT P-N-P in configurazione a emettitore comune}
\author{Enrico Barbuio \\ 0001117553 \and Giacomo Cicala \\ 0001122965} 
\date{\today}

\begin{document}
\maketitle

\renewcommand{\abstractname}{Abstract}
\begin{abstract}
    prova
\end{abstract}

\section*{Introduzione}

Equazione tensione di Early:
\begin{equation}
    \label{eq:VA}
    V_{A} = -\frac{a}{b}
\end{equation}

Si può ottenere inoltre una stima (\err{senza incertezza}) del guadagno di corrente
\begin{equation}
    \label{eq:beta}
    \beta = \frac{\Delta I_{C}}{\Delta I_{B}}
\end{equation}

\noindent
dove $\Delta I_{C}$ è la differenza tra le correnti di collettore misurate per le due correnti di base alla stessa tensione di collettore, e $\Delta I_{B}$ è la differenza tra le due correnti di base. 


\section*{Apparato sperimentale e svolgimento}

\section*{Risultati e discussione}
Si riportano in (\cref{tab:50uA}) e in (\cref{tab:100uA}) le misure delle coppie di valori I-V rispettivamente per corrente di base di $-50$ \mu A e $-100$ \mu A. I valori di tensione e corrente sono già stati cambiati di segno per ottenere le caratteristiche nel primo quadrante Per il calcolo degli errori associati alle misure di tensione e corrente si rimanda all'\cref{appendice:errori_I-V}.

\sisetup{
    table-number-alignment = center,
    table-figures-integer = 4,
    table-figures-decimal = 8
}

\begin{table}[hbtp]
    \centering
    % --- INIZIO PRIMA META' (SINISTRA) ---
    \begin{minipage}[t]{0.48\textwidth}
        \vspace{0pt}
        \centering
        \begin{tabular}[t]{
                S[table-format=1.3]
                S[table-format=2.2]
                S[table-format=1.3]
                S[table-format=1.2]
                S[table-format=1.1]
            }
            \toprule
            { \boldmath $-V_{CE}$ } & { \boldmath $-I_{C}$ } & { \boldmath $\sigma_{V}$ } & { \boldmath $\sigma_{I}$ } & { \boldmath $F.S._{osc}$ } \\
            { (V) }                & { (mA) }              & { (V) }                    & { (mA) }                   & { (V/div) }                \\
            \midrule
            4.00                   & 10.97                 & 0.16                       & 0.05                       & 1                          \\
            3.80                   & 10.92                 & 0.15                       & 0.05                       & 1                          \\
            3.60                   & 10.86                 & 0.15                       & 0.05                       & 1                          \\
            3.40                   & 10.80                 & 0.14                       & 0.05                       & 1                          \\
            3.20                   & 10.73                 & 0.14                       & 0.05                       & 1                          \\
            3.00                   & 10.62                 & 0.10                       & 0.05                       & 0.5                        \\
            2.80                   & 10.53                 & 0.10                       & 0.05                       & 0.5                        \\
            2.60                   & 10.46                 & 0.09                       & 0.05                       & 0.5                        \\
            2.40                   & 10.39                 & 0.09                       & 0.05                       & 0.5                        \\
            2.20                   & 10.32                 & 0.08                       & 0.05                       & 0.5                        \\
            2.00                   & 10.24                 & 0.08                       & 0.05                       & 0.5                        \\
            1.80                   & 10.14                 & 0.07                       & 0.05                       & 0.5                        \\
            1.60                   & 10.05                 & 0.07                       & 0.05                       & 0.5                        \\
            1.40                   & 9.97                  & 0.07                       & 0.04                       & 0.5                        \\
            1.20                   & 9.86                  & 0.06                       & 0.04                       & 0.5                        \\
            \bottomrule
        \end{tabular}
    \end{minipage}
    \hfill % Questo comando spinge le due minipage ai lati opposti
    % --- INIZIO SECONDA META' (DESTRA) ---
    \begin{minipage}[t]{0.48\textwidth}
        \centering
        % NOTA: Intestazioni ripetute qui per chiarezza
        \begin{tabular}[t]{
                S[table-format=1.3]
                S[table-format=2.2]
                S[table-format=1.3]
                S[table-format=1.2]
                S[table-format=1.1]
            }
            \toprule
            { \boldmath $-V_{CE}$ } & { \boldmath $-I_{C}$ } & { \boldmath $\sigma_{V}$ } & { \boldmath $\sigma_{I}$ } & { \boldmath $F.S._{osc}$ } \\
            { (V) }                & { (mA) }              & { (V) }                    & { (mA) }                   & { (V/div) }                \\
            \midrule
            1.00                   & 9.75                  & 0.04                       & 0.04                       & 0.2                        \\
            0.80                   & 9.62                  & 0.03                       & 0.04                       & 0.2                        \\
            0.60                   & 9.49                  & 0.03                       & 0.04                       & 0.2                        \\
            0.52                   & 9.41                  & 0.03                       & 0.04                       & 0.2                        \\
            0.44                   & 9.34                  & 0.02                       & 0.04                       & 0.2                        \\
            0.400                  & 9.27                  & 0.016                      & 0.04                       & 0.1                        \\
            0.360                  & 9.22                  & 0.015                      & 0.04                       & 0.1                        \\
            0.320                  & 9.15                  & 0.014                      & 0.04                       & 0.1                        \\
            0.280                  & 8.99                  & 0.013                      & 0.04                       & 0.1                        \\
            0.240                  & 8.63                  & 0.012                      & 0.04                       & 0.1                        \\
            0.200                  & 7.77                  & 0.012                      & 0.04                       & 0.1                        \\
            0.180                  & 7.08                  & 0.011                      & 0.04                       & 0.1                        \\
            0.160                  & 5.97                  & 0.011                      & 0.04                       & 0.1                        \\
            0.140                  & 4.81                  & 0.011                      & 0.04                       & 0.1                        \\
            \bottomrule
        \end{tabular}
    \end{minipage}
    \caption{Misure di tensione e corrente per $I_{B} = -50 \mu A$. La prima colonna riporta le tensioni collettore-emettitore, la seconda le correnti di collettore, la terza gli errori sulla misure di tensione, la quarta gli errori sulle misure di corrente, la quinta il fondo scala dell'oscilloscopio.}

    \label{tab:50uA}
\end{table}

\begin{table}[h]
    \centering
    % --- META' SINISTRA (Righe 1-32) ---
    \begin{minipage}[t]{0.48\textwidth}
        \vspace{0pt} % Trucco per allineamento
        \centering
        \begin{tabular}[t]{
                S[table-format=1.3] % V_CE
                S[table-format=2.2] % I_C (serve 2.2 per il 20.xx)
                S[table-format=1.3] % err V (serve 1.3 per i 0.00x)
                S[table-format=1.2] % err I
                S[table-format=1.1] % F.S.
            }
            \toprule
            { \boldmath $-V_{CE}$ } & { \boldmath $-I_{C}$ } & { \boldmath $\sigma_{V}$ } & { \boldmath $\sigma_{I}$ } & { \boldmath $F.S._{osc}$ } \\
            { (V) }                & { (mA) }              & { (V) }                & { (mA) }                    & { (V/div) }                   \\
            \midrule
            4.00                   & 20.68                 & 0.16                       & 0.06                       & 1                          \\
            3.80                   & 20.65                 & 0.15                       & 0.06                       & 1                          \\
            3.60                   & 20.56                 & 0.15                       & 0.06                       & 1                          \\
            3.40                   & 20.44                 & 0.14                       & 0.06                       & 1                          \\
            3.20                   & 20.31                 & 0.14                       & 0.06                       & 1                          \\
            3.00                   & 20.15                 & 0.13                       & 0.06                       & 1                          \\
            2.90                   & 19.89                 & 0.10                       & 0.06                       & 0.5                        \\
            2.80                   & 19.81                 & 0.10                       & 0.06                       & 0.5                        \\
            2.70                   & 19.70                 & 0.10                       & 0.06                       & 0.5                        \\
            2.60                   & 19.61                 & 0.09                       & 0.06                       & 0.5                        \\
            2.50                   & 19.54                 & 0.09                       & 0.06                       & 0.5                        \\
            2.40                   & 19.44                 & 0.09                       & 0.06                       & 0.5                        \\
            2.30                   & 19.32                 & 0.09                       & 0.06                       & 0.5                        \\
            2.20                   & 19.21                 & 0.08                       & 0.06                       & 0.5                        \\
            2.10                   & 19.12                 & 0.08                       & 0.06                       & 0.5                        \\
            2.00                   & 19.02                 & 0.08                       & 0.06                       & 0.5                        \\
            1.90                   & 18.88                 & 0.08                       & 0.06                       & 0.5                        \\
            1.80                   & 18.80                 & 0.07                       & 0.06                       & 0.5                        \\
            1.70                   & 18.70                 & 0.07                       & 0.06                       & 0.5                        \\
            1.60                   & 18.61                 & 0.07                       & 0.06                       & 0.5                        \\
            1.50                   & 18.48                 & 0.07                       & 0.06                       & 0.5                        \\
            1.40                   & 18.38                 & 0.07                       & 0.06                       & 0.5                        \\
            1.30                   & 18.26                 & 0.06                       & 0.06                       & 0.5                        \\
            1.20                   & 18.17                 & 0.06                       & 0.06                       & 0.5                        \\
            1.10                   & 18.04                 & 0.06                       & 0.06                       & 0.5                        \\
            1.00                   & 17.91                 & 0.04                       & 0.06                       & 0.2                        \\
            0.92                   & 17.82                 & 0.03                       & 0.06                       & 0.2                        \\
            0.84                   & 17.70                 & 0.03                       & 0.06                       & 0.2                        \\
            0.76                   & 17.60                 & 0.03                       & 0.06                       & 0.2                        \\
            0.68                   & 17.47                 & 0.03                       & 0.06                       & 0.2                        \\
            0.60                   & 17.31                 & 0.03                       & 0.06                       & 0.2                        \\
            0.52                   & 17.11                 & 0.03                       & 0.06                       & 0.2                        \\
            \bottomrule
        \end{tabular}
    \end{minipage}
    \hfill
    % --- META' DESTRA (Righe 33-64) ---
    \begin{minipage}[t]{0.48\textwidth}
        \vspace{0pt} % Trucco per allineamento
        \centering
        \begin{tabular}[t]{
                S[table-format=1.3]
                S[table-format=2.2]
                S[table-format=1.3]
                S[table-format=1.2]
                S[table-format=1.1]
            }
            \toprule
            { \boldmath $-V_{CE}$ } & { \boldmath $-I_{C}$ } & { \boldmath $\sigma_{V}$ } & { \boldmath $\sigma_{I}$ } & { \boldmath $F.S._{osc}$ } \\
            { (V) }                & { (mA) }              & { (V) }                & { (mA) }                    & { (V/div) }                   \\
            \midrule
            0.44                   & 16.80                 & 0.02                       & 0.06                       & 0.2                        \\
            0.40                   & 16.61                 & 0.02                       & 0.05                       & 0.2                        \\
            0.380                  & 16.10                 & 0.015                      & 0.05                       & 0.1                        \\
            0.340                  & 15.85                 & 0.014                      & 0.05                       & 0.1                        \\
            0.320                  & 15.60                 & 0.014                      & 0.05                       & 0.1                        \\
            0.300                  & 15.28                 & 0.013                      & 0.05                       & 0.1                        \\
            0.290                  & 15.05                 & 0.010                      & 0.05                       & 0.05                       \\
            0.280                  & 14.85                 & 0.010                      & 0.05                       & 0.05                       \\
            0.270                  & 14.60                 & 0.010                      & 0.05                       & 0.05                       \\
            0.260                  & 14.35                 & 0.009                      & 0.05                       & 0.05                       \\
            0.250                  & 14.04                 & 0.009                      & 0.05                       & 0.05                       \\
            0.240                  & 13.65                 & 0.009                      & 0.05                       & 0.05                       \\
            0.230                  & 13.27                 & 0.009                      & 0.05                       & 0.05                       \\
            0.220                  & 12.80                 & 0.008                      & 0.05                       & 0.05                       \\
            0.210                  & 12.30                 & 0.008                      & 0.05                       & 0.05                       \\
            0.200                  & 11.70                 & 0.008                      & 0.05                       & 0.05                       \\
            0.190                  & 11.07                 & 0.008                      & 0.05                       & 0.05                       \\
            0.180                  & 10.30                 & 0.007                      & 0.05                       & 0.05                       \\
            0.170                  & 9.49                  & 0.007                      & 0.04                       & 0.05                       \\
            0.160                  & 8.62                  & 0.007                      & 0.04                       & 0.05                       \\
            0.150                  & 7.75                  & 0.007                      & 0.04                       & 0.05                       \\
            0.140                  & 6.75                  & 0.007                      & 0.04                       & 0.05                       \\
            0.130                  & 5.76                  & 0.006                      & 0.04                       & 0.05                       \\
            0.120                  & 4.87                  & 0.006                      & 0.04                       & 0.05                       \\
            0.110                  & 4.07                  & 0.006                      & 0.04                       & 0.05                       \\
            0.100                  & 3.27                  & 0.004                      & 0.03                       & 0.02                       \\
            0.092                  & 2.68                  & 0.003                      & 0.03                       & 0.02                       \\
            0.084                  & 2.13                  & 0.003                      & 0.03                       & 0.02                       \\
            0.076                  & 1.68                  & 0.003                      & 0.03                       & 0.02                       \\
            0.068                  & 1.29                  & 0.003                      & 0.03                       & 0.02                       \\
            0.060                  & 0.96                  & 0.003                      & 0.03                       & 0.02                       \\
            \\
            \bottomrule
        \end{tabular}
    \end{minipage}
    \caption{Misure di tensione e corrente per $I_{B} = -100 \mu A$. La prima colonna riporta le tensioni collettore-emettitore, la seconda le correnti di collettore, la terza gli errori sulla misure di tensione, la quarta gli errori sulle misure di corrente, la quinta il fondo scala dell'oscilloscopio.}
    \label{tab:100uA}

\end{table}

Dalle caratteristiche di uscita del transistor per le due correnti di base riportate in (\cref{fig:fit_caratteristiche}), si sono eseguiti due fit lineari nella regione attiva del transistor $(1-3.5)$ V per ottenere il valore della tensione di Early.
\begin{equation}
    I_{C}= a+bV_{CE}
\end{equation}

\begin{figure}[hbtp]
    \centering
    \includegraphics[width=0.7\textwidth]{fit.pdf}
    \caption{Caratteristiche di uscita del transistor per le correnti di base $I_{B}=-50$ \mu A (in blu) e $I_{B}=-100$ \mu A (in rosso) con i relativi fit lineari nella regione attiva $(1-3.5)$ V. Sulle ascisse sono riportate le tensioni collettore-emettitore, sulle ordinate le correnti di collettore.}
    \label{fig:fit_caratteristiche}
\end{figure}

\noindent
I valori dei parametri di fit per le due correnti di base sono i seguenti:
\begin{equation*}
    a_{50} = (9.35 \pm 0.05) \text{ mV} \hspace{2cm} b_{50} = (0.43 \pm 0.02) \text{ mS}
\end{equation*}
\begin{equation*}
    a_{100} = (16.89 \pm 0.07) \text{ mV} \hspace{2cm} b_{100} = (1.06 \pm 0.03) \text{ mS}
\end{equation*}

\noindent
da cui, tramite \cref{eq:VA}, si ottengono i valori di tensione di Early
\begin{equation*}
    V_{A,50} = (-21.7 \pm 1.1) \text{ V} \hspace{2cm} V_{A,100} = (-16.0 \pm 0.5) \text{ V}
\end{equation*}

\noindent
le cui incertezze sono state calcolate come descritto nell'\cref{appendice:errori_VA}.

Tramite l' \cref{eq:beta}, scegliendo come tensione di collettore $-V_{CE} = 3$ V, si ottiene una stima del guadagno di corrente del transistor pari a
\begin{equation*}
    \beta = 113.267
\end{equation*}


Per quanto riguarda gli andamenti delle caratteristiche di uscita del transistor, si nota come per entrambe le correnti di base considerate, queste rispecchino qualitativamente la forma prevista dalle equazioni di Ebers-Moll per il BJT in configurazione a emettitore comune. Inoltre, si osserva che, in regione attiva, all'aumentare della corrente di base, aumenta linearmente anche la corrente di collettore, confermando l'effetto Early. Tuttavia, si nota come i valori della tensione di Early ottenuti dai fit lineari per le due correnti di base considerate non siano compatibili tra loro entro l'incertezza sperimentale.

\section*{Conclusioni}

I valori della tensione di Early ottenutiper le due correnti di base  a $-50$  \mu A e a $-100$  \mu A sono
\begin{equation*}
    V_{A,50} = (-21.7 \pm 1.1) \text{ V} \hspace{2cm} V_{A,100} = (-16.0 \pm 0.5) \text{ V}
\end{equation*}
\noindent
risultando non compatibili tra loro entro l'incertezza sperimentale.

Invece, la stima del guadagno di corrente del transistor ottenuta scegliendo come tensione di collettore $-V_{CE} = 3$ V è pari a
\begin{equation*}
    \beta = 113.267
\end{equation*}


\appendix
\section{Appendici}
\subsection{Calcolo degli errori per tensioni e correnti}
\label{appendice:errori_I-V}

La risoluzione dell'oscilloscopio (e quindi l'errore sulla lettura) $\sigma_{l}$ è stata calcolata come
\begin{equation}
    \sigma_{l} = \frac{F.S}{5}*(\text{\#tacchette apprezzabili})
\end{equation}
dove in questo caso, in tutte le misure dell'esperimento, il numero di tacchette apprezzabili è stato 0.5. L'errore totale associato ad ogni misura di tensione con l'oscilloscopio è stato calcolato come

\begin{equation}
    \sigma_{V} = \sqrt{(\sigma_{l})^{2} + (\sigma_{c})^{2}}
\end{equation}

\noindent
dove l'errore del costruttore è

\begin{equation}
    \frac{\sigma_{c}}{V_{mis}} = 3\%
\end{equation}

\noindent
con $V_{mis}$ tensione misurata; l'errore sullo zero dell'oscilloscopio è stato omesso in quanto si è verificato essere trascurabile rispetto agli altri errori, grazie ad un opportuno fondo scala di $5$mV/Div. Invece, l'errore sulla misura di corrente è stato calcolato considerando l'errore del multimetro, che, per il fondo scala utilizzato per tutte le misure $F.S._{mult} = 60$mA, è dato da
\begin{equation}
    \frac{\sigma_{I}}{I_{mis}} = 1.5\% + 3 \text{ digits}
\end{equation}

\subsection{Calcolo dell'errore sulla tensione di Early}
\label{appendice:errori_VA}
L'errore sulla tensione di Early è stato calcolato propagando gli errori sui parametri del fit lineare secondo la formula

\begin{equation}
    \sigma_{V_{A}} = V_{A} \sqrt{\left(\frac{\sigma_{a}}{a}\right)^{2} + \left(\frac{\sigma_{b}}{b}\right)^{2}}
\end{equation}


\end{document}